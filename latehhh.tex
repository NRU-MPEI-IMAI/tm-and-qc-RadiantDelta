\documentclass{article}
\usepackage[utf8]{inputenc}
\usepackage[T2A]{fontenc}
\usepackage[utf8]{inputenc}
\usepackage[russian]{babel}
\usepackage[margin=3cm]{geometry}
\usepackage{paralist}
\usepackage{amsthm, amsmath, amsfonts, amssymb}
\usepackage{mathtools} % \mathclap
\usepackage{bm}
\usepackage{dsfont}
\usepackage{hyperref}
\usepackage{graphicx}
\usepackage{multirow}
\usepackage{comment}
\usepackage{xcolor, colortbl}
\usepackage{xifthen, xspace}
\usepackage{caption, subcaption}
\usepackage{lscape}
\usepackage{braket}
\usepackage{epigraph}
\usepackage{sectsty}
\usepackage{listings}
\usepackage{graphicx}
\hypersetup{
    colorlinks=true,
    linkcolor=blue,
    filecolor=magenta,      
    urlcolor=blue,
    pdftitle={Overleaf Example},
    pdfpagemode=FullScreen,
    }

\title{Теоретические модели вычислений \\
ДЗ №3: Машины Тьюринга и квантовые вычисления}

\author{Игумнов Олег}

\date{1 мая 2022 года}

\begin{document}

\maketitle



\section{Машины Тьюринга}

Работу требуется выполнять в системе \url{turingmachine.io}. \\\\
Для сдачи заданий 1-2 требуется прикрепить файлы YAML с исходным кодом проекта. Каждый файлы должен иметь наименование задание\_пункт.yml, к примеру 1\_1.yml для первой задачи первого задания. \\\\

\subsection{Операции с числами}

Реализуйте машины Тьюринга, которые позволяют выполнять следующие операции:
\begin{enumerate}
    \item Сложение двух унарных чисел (1 балла)
    
    Алгоритм: заменяем плюс на 1, и удаляем последний символ
    
    \item Умножение унарных чисел (1 балл)
    
    Алгоритм: сначала ставим в конце флаг для результата (=), затем
    копируем число справа до тех пор пока не кончатся единицы левого числа
    
\end{enumerate}


\subsection{Операции с языками и символами}

Реализуйте машины Тьюринга, которые позволяют выполнять следующие операции:
\begin{enumerate}
    \item Принадлежность к языку $L = \{ 0^n1^n2^n \}, n \ge 0$ (0.5 балла)
    
    Алгоритм: учитываем случай n=0, будем ходить по строке сначала удаляя нули и заменяя единицы на символ а, затем, учитывая, что символов а столько же, сколько было нулей будем удалять символы а и заменять двойки на символ б, при этом будут совершаться проверки на соблюдение порядка следования символов
    
    \item Проверка соблюдения правильности скобок в строке (минимум 3 вида скобок) (0.5 балла)
    
    Алгоритм:учитываем случай пустого слова, ищем правую скобку, затем парную ей левую, заменяем обе на символ, возвращаемся в начало опять ищем правую скобку и тд.
    
    \item Поиск минимального по длине слова в строке (слова состоят из символов 1 и 0 и разделены пробелом) (1 балл)
    
    Алгоритм: поочередно заменяем символы в каждом слове, то слово, после которого будет обнаружен пробел и есть искомое, помечаем его и удаляем лишнее, после возвращаем ему нормальный вид
\end{enumerate}


\section{Квантовые вычисления}

Для выполнения заданий по квантовым вычислениям требуется QDK. Его можно скачать здесь: \url{https://docs.microsoft.com/en-us/azure/quantum/install-overview-qdk}. 
\\\\
Но можно использовать любой пакет, типа \url{https://qiskit.org/}. 
\\\\
В качестве решения задачи надо предоставить схему алгоритма для частного случая при фиксированном количестве кубитов и фиксированных состояниях. 


\subsection{Генерация суперпозиций 1 (1 балл)}

Дано $N$ кубитов ($1 \le N \le 8$) в нулевом состоянии $\Ket{0\dots0}$. Также дана некоторая последовательность битов, которое задаёт ненулевое базисное состояние размера $N$. Задача получить суперпозицию нулевого состояния и заданного.

Частный случай

$$\Ket{S} = \frac{1}{\sqrt2}(\Ket{000} +\Ket{101})$$



 см. рис 1. 2.1
 
код:
\begin{lstlisting}
operation Solve (qs: Qubit[], bits: Bool[]) : () 
{
    body 
    {
        H(qs[0]);  
        for (i in 1..Length(qs) - 1)
        {

            if (bits[i]) 
            {
                CNOT(qs[0], qs[i]);
            }
        }
    }
}

\end{lstlisting}

\begin{figure}[h]
\centering
\caption{2.1}
\includegraphics[scale=0.9]{31.png}

\end{figure}

\subsection{Различение состояний 1 (1 балл)}

Дано $N$ кубитов ($1 \le N \le 8$), которые могут быть в одном из двух состояний:

Пусть N=3

$$\Ket{GHZ} = \frac{1}{\sqrt2}(\Ket{000} +\Ket{111})$$
$$\Ket{W} = \frac{1}{\sqrt 3}(\Ket{100}+\Ket{010} +\Ket{001})$$

Требуется выполнить необходимые преобразования, чтобы точно различить эти два состояния. Возвращать $0$, если первое состояние и 1, если второе. 
\\\\
код:
\begin{lstlisting}
operation Solve (qs: Qubit[]) : Int
{
    body 
    {
        mutable countOnes = 0;
        for (i in 0..Length(qs) - 1)
        {
            if (M(qs[i]) == One)
            {
                set counter = counter + 1;
            }
        }
        if (counter == 1)
        {
            return 1;
        }
        return 0;
    }
}

\end{lstlisting}










\subsection{Различение состояний 2 (2 балла)}

Дано $2$ кубита, которые могут быть в одном из двух состояний:

$$\Ket{S_0} = \frac{1}{2}(\Ket{00} + \Ket{01} + \Ket{10} + \Ket{11})$$
$$\Ket{S_1} = \frac{1}{2}(\Ket{00} - \Ket{01} + \Ket{10} - \Ket{11})$$
$$\Ket{S_2} = \frac{1}{2}(\Ket{00} + \Ket{01} - \Ket{10} - \Ket{11})$$
$$\Ket{S_3} = \frac{1}{2}(\Ket{00} - \Ket{01} - \Ket{10} + \Ket{11})$$


Требуется выполнить необходимые преобразования, чтобы точно различить эти четыре состояния. Возвращать требуется индекс состояния (от $0$ до $3$). 
\\\\
код:
\begin{lstlisting}
operation Solve (qs: Qubit[]) : Int
    {
        body 
        {
			H(qs[0]);
			H(qs[1]);
			if (M(qs[0]) == Zero) 
			{
				if (M(qs[1]) == Zero)
				{
					return 0;
				}
				else
				{
					return 1;
				}
			}
			else
			{
				if (M(qs[1]) == Zero)
				{
					return 2;
				}
				else
				{
					return 3;
				}
			}
        }
    }

\end{lstlisting}


\subsection{Написание оракула 1 (2 балла)}

Требуется реализовать квантовый оракул на $N$ кубитах ($1 \le N \le 8$), который реализует следующую функцию: $f(\pmb{x}) = (\pmb{b}\pmb{x}) \mod 2$, где  $\pmb{b} \in \{0,1\}^N$ вектор битов и  $\pmb{x}$ вектор кубитов. Выход функции записать в кубит $\pmb{y}$. Количество кубитов $N$ ($1 \le N \le 8$). 
\\\\
код:
\begin{lstlisting}
operation Solve (x: Qubit[], y: Qubit, b: Int[]) : ()
{
    body 
    {
        Reset(y);
        for (i in 0..Length(x) - 1)
        {
            if (b[i] == 1) {
                CNOT(x[i], y);
            }
        }
    }
}

\end{lstlisting}

схема для входных данных [1 0 1 1]



\begin{figure}[h]
\centering
\caption{2.4}
\includegraphics[scale=0.9]{34.png}

\end{figure}

\end{document}